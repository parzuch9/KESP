\documentclass[10pt,showpacs,preprintnumbers,footinbib,amsmath,amssymb,aps,prl,twocolumn,groupedaddress,superscriptaddress,showkeys]{revtex4-1}
\usepackage{graphicx}
\usepackage{dcolumn}
\usepackage{bm}
\usepackage[colorlinks=true,urlcolor=blue,citecolor=blue]{hyperref}
\usepackage{color}
\begin{document}
\title{Project 1}
\author{Parker Brue}
\affiliation{Department of Physics and Astronomy, Michigan State University, East Lansing, MI 48823}
\begin{abstract}
We present our Ferrari algorithm for solving linear equations. We wrote the one-dimensional Poisson equation, utilizing Dirichlet boundary conditions, as a linear set of equations and as a tridiagonal matrix. We compared a specialized algorithm for solving the tridiagonal matrix to an LU-decomposition of said matrix.  Our best algorithm, the specialized solver, runs as $4n$ FLOPS with $n$ the dimensionality of the matrix.
\end{abstract}
\maketitle

\section{Introduction}
As an introduction to the central ideas of the class, we studied the one-dimensional Poisson equation with Dirichlet boundary conditions. Namely, transforming the differential equation into a set of linear equations, and consequently, a matrix. We implemented a general algorithm, a specialized algorithm, and a LU-decomposition algorithm. In the course of this report, we introduce the theoretical model and the different algorithms  we developed, then discuss the results of the different methods. Important points of comparison lie with relative error and relative speed of the calculation due to FLOPS. 

\section{Theory, algorithms and methods}
	\subsection{Theoretical solution of the one dimensional Poisson equation}	
In general, The one dimensional Poisson equation reads as follows: \begin{equation} -u^{''}(x) = f(x)     \end{equation}  Through discretized approximation of {\it u}, we can solve for {\it f} using a set of linear equations: \begin{equation}
	f(x)=-\frac{v_{i+1}+u_{i-1}-2u_{i}}{h^{2}}      ;      i=1,...,n
	\end{equation}
Using Dirichlet boundary conditions, {\it u}\textsubscript{0} = {\it u}\textsubscript{n+1} = 0,  We can then rewrite this as a set of linear equations in the form of a tridiagonal matrix: \begin{equation}	
	\hat{A} \cdot \hat{u} = \hat{f}
	\end{equation}
Consequently, we can solve these linear equations through forward and backward substitution.

	\subsection{Specific Poisson equation}	
For our purposes of solving the Poisson equation, we assume a function \begin{equation}
	f(x)=100e^{-10x}
	\end{equation}
and a closed form solution with the Dirichlet boundary conditions:
	\begin{equation}
	u(x)=1-(1-e^{-10})x-e^{-10x}
	\end{equation}
We can write a simple program that uses forward (\ref{forward}) and backward (\ref{backward}) substitutions to solve the set of linear equations.
	
	\begin{equation}
	\tilde{f}_{i}=f_{i}-\frac{\tilde{f}_{i-1}e_{i-1}}{\tilde{d}_{i-1}} ; \tilde{d}_{i}=d_{i}-e^{2}_{i-1}/\tilde{d}_{i-1}
	\label{forward}
	\end{equation}
	
	\begin{equation}
	u_{i} = (\tilde{f}_{i}-e_{i}u{i+1})/\tilde{d}_{i}
	\label{backward}
	\end{equation}
where $d_{i}$ are the diagonal matrix elements and $e_{i}$ are the off diagonal matrix elements.
	


We were able to create another program that performed matrix mathematics to solve for the set of linear equations.  Taking a tridiagonal matrix, we can solve it using LU decomposition (Figure 1\ref{ludecomp}).
	
	
	\begin{figure}[!ht]
			\centering
			
			\label{ludecomp}
			\caption{The code to find the inverse of a matrix utilizing LU decomposition}
		\end{figure}
	

	\subsection{Relative error}
In order to properly test the effectiveness of our algorithm, we are measuring the relative closeness, or relative error of our analytic solution:
	\begin{equation}
	\epsilon_{i} = log_{10}(|\frac{v_{i}-u_{i}}{u_{i}}|)  ; i = 1,...,n
	\label{error}
	\end{equation}
{\it u}\textsubscript{i} is the analytic solution, and {\it v}\textsubscript{i}

\section{Results and discussions}

\begin{figure}[hbtp]

\caption{Exact and numerial solutions for $n=10$ mesh points.} 
\label{fig:n10points}
\end{figure}

\section{Conclusions}

\begin{thebibliography}{99}
\bibitem{miller2006} G.~A.~Miller, A.~K.~Opper, and E.~J.~Stephenson, Annu.~Rev.~Nucl.~Sci.~{\bf 56}, 253 (2006).
\end{thebibliography}

\end{document}