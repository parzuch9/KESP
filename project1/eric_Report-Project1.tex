\documentclass[10pt,showpacs,preprintnumbers,footinbib,amsmath,amssymb,aps,prl,twocolumn,groupedaddress,superscriptaddress,showkeys]{revtex4-1}

\usepackage{graphicx}
\usepackage{dcolumn}
\usepackage{bm}
\usepackage[colorlinks=true,urlcolor=blue,citecolor=blue]{hyperref}
\usepackage{color}




\begin{document}
	\title{Project 1}
	\author{Eric Aboud}
	\affiliation{Department of Physics and Astronomy, Michigan State University, East Lansing, MI 48823}
	\begin{abstract}
		We present our algorithm for solving linear equations.  Using Dirichlet boundary conditions, we were able to write the one-dimensional Poisson equation using a set of linear equations. Our best algorithm runs as $4n$ FLOPS with $n$ the dimensionality of the matrix.  We were then able to rewrite our linear set of equations as a tridiagonal matrix, solving the equations.  Comparions between the two methods provides insight on the optimal way of solving the set of linear equations in C++.
	\end{abstract}
	\maketitle


	
	
	\section*{Introduction}
	In order to become more familiar with vector and matrix programming, this project was set up to write matrix operations and the linear operations behind them.  Using Dirichlet boundary conditions, we were able to write a set of linear equations to solve the one dimensional Poisson equation.  We were also able write a linear set of equations to form a tridiagonal matrix.
	
	
	
	\section*{Theory, algorithms and methods}
	
	\subsection{Theoretical solution of the one dimensional Poisson equation}
	
	We can solve the one dimensional Poisson equation by setting a function equal to the negative of the second derivative of another function: \begin{equation} -u^{''}(x) = f(x)     \end{equation}  Via descritization, we can solve for {\it f} using a set of linear equations: \begin{equation}
	f(x)=-\frac{v_{i+1}+u_{i-1}-2u_{i}}{h^{2}}      ;      i=1,...,n
	\end{equation}
	Using Dirichlet boundary conditions, we are able to solve the set of linear equations using forward and backward substitutions.  We can then rewrite this as a set of linear equations in the form of a tridiagonal matrix: \begin{equation}	
	\hat{A} \cdot \hat{u} = \hat{f}
	\end{equation}
	
	
	\subsection{Computational solution of the one dimensional Poisson equation}
	
	It is possible to solve the poisson equation using a set of linear equations and matrix mathematics with C++.  By assume a function \begin{equation}
	f(x)=100e^{-10x}
	\end{equation}
	and a closed form solution:
	\begin{equation}
	u(x)=1-(1-e^{-10})x-e^{-10x}
	\end{equation}
	We can write a simple program that uses forward (\ref{forward}) and backward (\ref{backward}) subsitutions to solve the set of linear equations.
	
	\begin{equation}
	\tilde{f}_{i}=f_{i}-\frac{\tilde{f}_{i-1}e_{i-1}}{\tilde{d}_{i-1}} ; \tilde{d}_{i}=d_{i}-e^{2}_{i-1}/\tilde{d}_{i-1}
	\label{forward}
	\end{equation}
	
	\begin{equation}
	u_{i} = (\tilde{f}_{i}-e_{i}u{i+1})/\tilde{d}_{i}
	\label{backward}
	\end{equation}
	
	where $d_{i}$ are the diagonal matrix elements and $e_{i}$ are the off diagonal matrix elements.
	
	We were able to create another program that performed matrix mathematics to solve for the set of linear equations.  Taking a tridiagonal matrix, we can solve it using LU decomposition (Figure 1\ref{ludecomp}).
	
	
	\begin{figure}[!ht]
			\centering
			\includegraphics[width=0.5\textwidth]{/Users/ericaboud/Pictures/PHY480-2.jpg}
			\label{ludecomp}
			\caption{The code to find the inverse of a matrix utilizing LU decomposition}
		\end{figure}
	
	We are able to calculate the relative error of the calculations using:
	\begin{equation}
	\epsilon_{i} = log_{10}(|\frac{v_{i}-u_{i}}{u_{i}}|)  ; i = 1,...,n
	\label{error}
	\end{equation}

	
	
	
	
	\section*{Results and discussion}
	
  By running the program for different step sizes with x in a range [0,1], we can see how the data changes with data points (Figure 2\ref{uvx}).    We can see from the plot, that without enough data points we cannot produce the full curve, but with too many data points we recieve a larger error.  This provides us with the knowledge that there is a saddle point in the step size where we optimize both uncertainty and completeness, in our case it appears to be the N=100 or N=$10^{2}$ step size.

\begin{figure}[!ht]
	\centering
	\includegraphics[width=0.5\textwidth]{/Users/ericaboud/Pictures/uvx.jpg}
	\label{uvx}
	\caption{Comparison between different step sizes.  The x axis are the x values in the range [0,1] and the y axis are the resulting u values.}
\end{figure}

We can further optimize our program by solving it for a special case, in which the diagonal matrix elements are equal to 2 and the off diagonal matrix elements are equal to -1.  This allows us to reduce the number of floating point operations per second (FLOPS) from an original 9 down to 4.  This allows us to run our program much faster by precalculating coefficients.




\begin{center}
	\begin{tabular}{cc}
		\hline \hline
			Step Size ($n$) &  Average Relative Error ($|\epsilon|$)\\
			\hline
			1 & 1.101\\
			2 & 3.079\\
			3 & 5.079\\
			4 & 7.079\\
			5 & 9.079\\
			6 & 11.50\\
			7 & 12.27\\
			\hline
			\label{errortable}
	\end{tabular}
\end{center}
	
	By including equation \ref{error} in our code, we can calculate the relative error for a certain step size.  The was done for multiple step sizes, $N=10^{n}$ n=1,...,7 (Table above).    It should be noted that these are the absolute values and averages of the errors.  It should also be noted that it breaks down for n=7, as there is a large fluctuation in the error between 10 and infinity.  It appears that n=6 is the largest order of step sizes that we can accurately use to describe the solutions to the function.
	
	We may also include a timer to calculate the time required to compute the solutions.  By adding a timing mechanism, we calculated the time required to solve square matrices with various column lengths.
	
	\begin{center}
		\begin{tabular}{cc}
			\hline \hline
			Column Size & Approximate Time (s)\\
			\hline
			10 & 0\\
			100 & 0\\
			1000 & 30\\
			10000 & \\
			1000000 & \\
			\hline
			\label{timingtable}
		\end{tabular}
	\end{center}
	
	
	
	\section*{Conclusion}
	
	
	
	
	
	
	
	
	
	
	
	
	
	
\end{document}